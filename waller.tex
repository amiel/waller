\documentclass[11pt]{report}

	\usepackage[style=verbose-note,bibstyle=reading,annotation=true]{biblatex}	
	\bibliography{bibliography}
	
	\usepackage{graphicx}
	\usepackage{amsmath}
	
	\usepackage{fancyheadings}
	\usepackage{setspace}
	

	\setstretch{1.7}
	% \usepackage{fullpage}
	
	\bibliography{bibliography}
	\makeatletter
	\renewcommand\@biblabel[1]{}
	\makeatother
	
	
	% \newcommand{\footnoteremember}[2]{\footnote{#2}\newcounter{#1}\setcounter{#1}{\value{footnote}}}
	% \newcommand{\footnoterecall}[1]{\footnotemark[\value{#1}]}
	
	
	\pagestyle{fancy}
	\rhead{Amiel Martin}
	\lhead{Fats Waller}
	
	
	\title{Fats Waller}
	\author{Amiel Martin}	
	\date{\today}
	
\begin{document}
	\maketitle
	
	
	% \section{Introduction} % (fold)
	\label{sec:introduction}
	
		When Fats Waller was introduced to Billy Kyle, pianist for John Kirby, Waller related a recording of theirs he had heard recently. Patting Kyle on the back he said, ``That modulation you play from A-flat to F was simply terrific! Where did you get that one from, man?'' To which Kyle replied; ``I took it off a record of yours.''\footnote{\cite[231]{anecdotes}}
		Fats Waller had more skill than he is attributed with. Because he lived in the moment, he became successful with his ``style'' in order to support his larger-than-life life.
		
		Known for his satirical lyrics and Harlem Stride style at the piano, Fats Waller was a more skilled musician than he gets credit for. He used his satirical lyrics to poke fun at mediocrity in music instead of treating it with spite. Or, as Morroe Berger points out in the Journal of Jazz Studies, Fats Waller's ``clowning and gluttony were only transparent covers for his artistic disappointment in America''\footnote{\cite[4]{outside-insider}}
		
		
		``The only music that he played without ever mocking it was the spiritual, which he recorded beautifully but infrequently on the organ.''\footnote{\cite[8]{outside-insider}}
		
		``Many observers have asserted that Waller's clowning and gluttony were only transparent covers for his artistic disappointment in America''\footnote{\cite[4]{outside-insider}} ``As a performer in show-biz, hovwever, he carefully concealed such discontents as may have rankled him and usually revealed only the good nature and good cheer that most audiences still expect of their entertainers.''\footnote{\cite[4]{outside-insider}}
		
		He did not try to make his audience ``feel inferior to the music'' like some jazz artists such as Miles Davis and Charlie Mingus.\footnote{\cite[16]{outside-insider}}
		
		
		
		
	
	% section introduction (end)
	
	% \printbibliography
	
\end{document}