\documentclass[11pt]{report}

	\usepackage[myheadings]{fullpage}


	\usepackage[style=verbose-note,bibstyle=authortitle]{biblatex}
	\bibliography{bibliography}

	\usepackage{graphicx}
	\usepackage{amsmath}
	\usepackage{subfigure}
	\subfiguretopcaptrue

	\usepackage{fancyhdr}
	\usepackage{setspace}
	\setstretch{1.7}

	\bibliography{bibliography}
	\makeatletter
	\renewcommand\@biblabel[1]{}
	\makeatother

	\renewcommand{\figurename}{Example}

	% \newcommand{\footnoteremember}[2]{\footnote{#2}\newcounter{#1}\setcounter{#1}{\value{footnote}}}
	% \newcommand{\footnoterecall}[1]{\footnotemark[\value{#1}]}

	\def\fattitle{``One never knows -- do one?''}
	\def\fatslogan{the skills of Fats Waller}




	\pagestyle{fancy}
	\rhead{Amiel Martin}
	\lhead{\fattitle{}: \fatslogan}


	\title{\fattitle \\
	\large \fatslogan}
	\author{Amiel Martin}	
	\date{\today}

\usepackage{graphics}
\begin{document}
	\maketitle

	\label{sec:introduction}
	When Fats Waller was introduced to Billy Kyle, pianist for John Kirby, Waller related a recording of theirs he had heard recently. Patting Kyle on the back he said, ``That modulation you play from A-flat to F was simply terrific! Where did you get that one from, man?'' To which Kyle replied, ``I took it off a record of yours.''\footnote{\cite[231]{anecdotes}.} This story, along with many others from \emph{Jazz Anecdotes}, demonstrates a consequence of his impetuous and larger-than-life lifestyle; he very much lived in the moment. In order to support this outsized lifestyle in financially difficult times, Waller turned to comedy, popular music, and high volume. He was best known for his satirical lyrics and stride style at the piano -- even so, Fats Waller was a more skilled musician than he gets credit for.

	\label{sec:piano_skills}
	Thomas Wright ``Fats'' Waller is known as ``the greatest of the Harlem jazz pianists''\footnote{\cite[2]{life}.} and the only stride pianist to become famous because of it.\footnote{\cite[146]{visions}.} Fats had a large range of stylistic capabilities, but by far the most popular was his stride. The term ``stride'' refers to a style of piano playing originated by James P. Johnson and probably derived from the ragtime technique of alternating between low notes and higher chords with the left hand.\footnote{for more about ``stride'', see \cite[79]{experience}.} In \emph{Fats Waller: his Life \& Times}, Alyn Shipton explains that as stride diverged from ragtime, the left-hand parts became more complex and less regular, and the right-hand parts became more complex and difficult as well.\footnote{\cite[5]{life}.} Fats picked up many techniques and prowess from Johnson as his mentor and by watching him perform cutting-contests (a form of musical battle in the 1920s-1930s). During that time he also picked up from Johnson ``a love of the classics, a respect for musical literacy, and (so far as the piano was concerned) a taste and delicacy of touch that allowed Fats the full dynamic range in his playing.''\footnote{\cite[8]{life}.}
	
	Waller's stride technique, as he puts it, is ``all in knowing what to put on the right beat''\footnote{quote from \emph{Metronome} 52:2 (February 1936): 33 found in \cite{transcriptions}.}; a steady pulsating bass in the left hand, embellishments in the right hand that are sparse and subservient to the melody, and overall, expression through rising and falling with sudden contrasts to build climaxes.
	
	
		\begin{figure}[ht]
			\centering
			\begin{minipage}{\textwidth}
				
				\caption{excerpt from \emph{My feelin's are hurt} (1929) transcr. B. Dobbens.\protect\footnote{\scriptsize From \cite[]{grove-book:waller}.}}
				\label{fig:hurt}
				

				{%
\parindent 0pt
\ifx\preLilyPondExample \undefined
\else
  \expandafter\preLilyPondExample
\fi
\def\lilypondbook{}%
\input b4/lily-d03c2438-systems.tex
\ifx\postLilyPondExample \undefined
\else
  \expandafter\postLilyPondExample
\fi
}


			\end{minipage}
		\end{figure}

	Fats was exceptional at improvising over classic stride. He also augmented the standard 2 beat stride rhythm with occasional three-beat cross-rhythms. These rhythms are exemplified in Example \ref{fig:hurt}, along with many of his other standard style traits, such as his tasteful variety of tone and texture (a skill he probably developed as an organ player), and the chromatic alterations and added inner pitches to octaves and 10\textsuperscript{ths} in the left hand that were undoubtedly a large influence to Art Tatum (another stride pianist, and one of the few who could rival Waller's mastery in technique).\footnote{\cite[40]{grove-book:waller}.} Art Tatum wasn't alone in looking up to Fats Waller. His creative style influenced other notable artists such as Count Basie and Dizzy Gillespie.\footnote{Waller's friendship and influence on both Count Basie and Art Tatum is mentioned in many sources. How he influenced Gillespie came from an interview with Gillespie on the CBS network program ``60 Minutes'' (which I discovered in \citetitle{transcriptions} by \cite{transcriptions}).}



	\label{sec:organ_and_classical}
	The Waller family acquired their first piano when Fats was merely 6 and he showed interest in it immediately,\footnote{See \cite{transcriptions} and \cite{life}.} and soon learned to read music at a very early age. Although he was truly gifted at the piano, the full breadth of Waller's skill was with the organ. He holds the title as the first significant organist in jazz.\footnote{\cite[40]{grove-book:waller}.} His interest in the organ could be attributed to the fact that his father was a clergyman. However, to his fathers dislike, early in his career, Waller took his skills to the theater. He accompanied silent movies at the Lafayette and Lincoln theaters on the pipe organ and also played for vaudeville shows. While most of the music of Fats Waller was satirical, he did have a serious side. It was spiritual music, which he played beautifully on the organ and was the only music he played without mocking it.\footnote{\cite[8]{outside-insider}.} Francis Newton also explains his work on the organ in \emph{The Jazz Scene}:
	
	\begin{quote}
		\ldots Fats Waller, a musician of sound classical training and superb technique, whose favourite instrument was the organ, and whose main ambition was to excel as an interpreter of J.S. Bach's ecclesiastical works, was never given the chance to record in this field.\footnote{\cite[209]{jazz_scene}.}
	\end{quote}
	
	Although not as publicly known, Fats also played and studied the music of other great classical and romantic European composers. There is no recorded evidence of Waller playing European piano music, but there is evidence in the form of an interview with Andy Razaf, a longtime collaborator with Waller\footnote{Andy wrote lyrics for some of Waller's most popular songs such as \emph{Ain't Misbehavin'}.}: ``He knew Brahms, Liszt and Beethoven as well as he knew jazz, and often discussed and analyzed their work.''\footnote{another quote from \emph{Metronome} (Andy Razaf, ``Fats Waller,'' \emph{Metronome} 60:1 (Janurary 1944): 16) found in \citetitle{transcriptions} by \cite{transcriptions}.}
	
	Fats Waller's grasp of classical music is apparent in his melodic embellishments and his proficiency with harmony and chordal structure. In 1941 he recorded a rendition of \emph{Honeysuckle Rose} that uniquely illustrates is familiarity with classical music.\footnote{\cite{transcriptions}} In this recording, entitled ``Honeysuckle Rose, \`a la Bach, Beethoven, Brahms, and Waller,'' he reverses the standard of reworking classics into jazz by reworking one of his own pieces to a classical style. He displays a wide variety of styles that \citeauthor{transcriptions} compares to specific passages from J.S. Bach and Brahms. Example \ref{fig:3bs} summarizes one of these comparisons.\footnote{For more, see \cite{transcriptions}.}

		\begin{figure}[ht]
			\centering
					\begin{minipage}{\textwidth}
				
				\caption{melodic comparison between Waller and J.S. Bach \protect\footnote{\scriptsize these transcriptions and similar comparisons are from \citeauthor{transcriptions}, see n. 7}}
				\label{fig:3bs}
				
				\subfigure[Fats Waller: \emph{Honeysuckle Rose, \`a la Bach, Beethoven, Brahms, and Waller} mm. 43--44]{

				{%
\parindent 0pt
\ifx\preLilyPondExample \undefined
\else
  \expandafter\preLilyPondExample
\fi
\def\lilypondbook{}%
\input 87/lily-f22dd579-systems.tex
\ifx\postLilyPondExample \undefined
\else
  \expandafter\postLilyPondExample
\fi
}

				}



							\subfigure[J.S. bach: Concerto in D Minor, BWV 1052, first movement, mm. 28--29]{


							{%
\parindent 0pt
\ifx\preLilyPondExample \undefined
\else
  \expandafter\preLilyPondExample
\fi
\def\lilypondbook{}%
\input 23/lily-5572ed49-systems.tex
\ifx\postLilyPondExample \undefined
\else
  \expandafter\postLilyPondExample
\fi
}

							}
						\end{minipage}

		\end{figure}
	
	Through these examples, it is quite clear that when Waller subtitles his piece ``\`a la Bach, Beethoven, Brahms, and Waller'', he knows what he is talking about.

	\label{sec:charisma}
	Probably his most lucrative skill, however, was bridging the gap between the artist and the public. Along with Louis Armstrong, he was one of the first to do this.\footnote{\cite[3]{life}} According to \citeauthor{outside-insider}, some jazz artists at the time such as Miles Davis and Charlie Mingus made their audience ``feel inferior to the music''.\footnote{\cite[16]{outside-insider}.} Fats did not do this. He gained popularity with humor and charisma even though the general public may not have understood his technical genius. It is easy to see how he accomplished this by watching him play.\footnote{All four of his `soundies' can be found on youtube \cite{youtube-joint_is_jumpin}.} His friendly voice and snide remarks are enough to entertain. However, with his expressive eyebrows, jumpy eyes, and dynamic jaw, his expressions could easily entertain the deaf. His remarkable piano playing is almost inconspicuous behind his expressive personality.
		
	\label{sec:respect}
	
	On the other hand, like his mentor James P. Johnson, he did have a profound respect for musical literacy and probably used satire to mask his true feelings about American popular music. In \emph{The Outside Insider}, Morroe Berger effectively explains this relationship:
	\begin{quote}
		Many observers have asserted that Waller's clowning and gluttony were only transparent covers for his artistic disappointment in America. As a performer in show-biz, hovwever, he carefully concealed such discontents as may have rankled him and usually revealed only the good nature and good cheer that most audiences still expect of their entertainers.\footnote{\cite[4]{outside-insider}}
	\end{quote}

	He used his satirical lyrics to poke fun at mediocrity in music instead of treating it with spite. Or, as Morroe Berger points out in the Journal of Jazz Studies, Fats Waller's ``a''\footnote{\cite[4]{outside-insider}}


	\label{sec:conclusion}

	``Unfortunately, Waller's public often demanded more of his exaggerated stage personality than of his unique creative gifts''\footnote{\cite[40]{grove-book:waller}}

	lived a short life, what would have happened if had lived longer / focused and recorded more academicly? ``one never knows -- do one?''



		\nocite{anecdotes}
		\nocite{experience}
		\nocite{grove-book:waller}
		\nocite{jazz_scene}
		\nocite{life}
		\nocite{outside-insider}
		\nocite{transcriptions}
		\nocite{visions}
		\nocite{web:machlin}
		\nocite{web:stride}
		\nocite{youtube-joint_is_jumpin}
	\printbibliography

\end{document}